\documentclass{article}
\usepackage{amsmath, amssymb, xpatch, amsthm, outlines, mathrsfs}
%\usepackage{mathcal}
\usepackage[utf8]{inputenc}

\theoremstyle{plain}

\makeatletter
\xpatchcmd{\@thm}{\thm@headpunct{.}}{}{}

%\newtheorem*{two_one_one_defa*}{Definition 2.1.1 (Supremum)}
%\newtheorem*{two_one_one_defb*}{Definition 2.1.1 (Infimum)}
%\newtheorem*{two_one_two_definition*}{Definition 2.1.2}
%\newtheorem*{two_one_three_definition*}{Definition 2.1.3}
%
%\newtheorem*{two_one_one_theorem*}{Theorem 2.1.1}
%\newtheorem*{two_one_two_theorem*}{Theorem 2.1.2}
%
%\newtheorem*{two_one_three_theorem*}{Theorem 2.1.3}
%
%
%\newtheorem*{two_one_four_theorem*}{Theorem 2.1.4}
%\newtheorem*{two_one_five_theorem*}{Theorem 2.1.5}
%\newtheorem*{two_one_six_theorem*}{Theorem 2.1.6}
%\newtheorem*{two_one_seven_theorem*}{Theorem 2.1.7}
%
%
%\newtheorem*{lemma221*}{Lemma 2.2.1}
%\newtheorem*{theorem221*}{Theorem 2.2.1}
%%\newtheorem*{corollary221*}{Corollary 2.2.1}
%
%\newtheorem*{theorem222*}{Theorem 2.2.2}
%\newtheorem*{theorem223*}{Theorem 2.2.3}
%\newtheorem*{theorem224*}{Theorem 2.2.4}
%
%
%\newtheorem*{corollary231*}{Corollary 2.3.1}
%\newtheorem*{theorem231*}{Theorem 2.3.1}
%\newtheorem*{theorem232*}{Theorem 2.3.2}
%\newtheorem*{theorem233*}{Theorem 2.3.3}
%
%
%%\newtheorem*[definition241*}{Definition 2.4.1}
%\newtheorem*{definition241*}{Definition 2.4.1}
%\newtheorem*{definition242*}{Definition 2.4.2}
%
%\newtheorem*{theorem241*}{Theorem 2.4.1}
%\newtheorem*{theorem242*}{Theorem 2.4.2}
%\newtheorem*{theorem243*}{Theorem 2.4.3}
%\newtheorem*{theorem244*}{Theorem 2.4.4}
%
%\newtheorem*{def151*}{Definition 1.5.1}
%\newtheorem*{theorem151*}{Theorem 1.5.1}
%\newtheorem*{def152*}{Definition 1.5.2}
%
%\newtheorem*{def111*}{Definition 1.1.1 Countable and Uncountable}
%

\newtheorem*{def231*}{Definition 2.3.1 Open Cover of a Set}
\newtheorem*{def232*}{Definition 2.3.2 Compact Set}

\newtheorem*{def418*}{Definition 4.18 Uniformly Continuous Function}
\newtheorem*{theorem419*}{Theorem 4.19}

\newtheorem*{theorem423*}{Theorem 4.23 Intermediate Value Theorem}

\newtheorem*{def61*}{Definition 6.1 Partition of a Closed Real Interval}
\newtheorem*{def62*}{Definition 6.2}
\newtheorem*{def63*}{Definition 6.3 Refinement of a Partition}
\newtheorem*{theorem64*}{Theorem 6.4}
\newtheorem*{theorem65*}{Theorem 6.5}
\newtheorem*{theorem66*}{Theorem 6.6}
\newtheorem*{theorem67*}{Theorem 6.7}
\newtheorem*{theorem68*}{Theorem 6.8}
\newtheorem*{theorem69*}{Theorem 6.9}
\newtheorem*{theorem610*}{Theorem 6.10}
\newtheorem*{theorem611*}{Theorem 6.11}
\newtheorem*{theorem612*}{Theorem 6.12}
\newtheorem*{theorem613*}{Theorem 6.13}
\newtheorem*{def614*}{Definition 6.14 Unit Step Function}
\newtheorem*{theorem615*}{Theorem 6.15}
\newtheorem*{theorem616*}{Theorem 6.16}
\newtheorem*{theorem617*}{Theorem 6.17}
\newtheorem*{theorem619*}{Theorem 6.19 (Change of Variables)}
\newtheorem*{theorem620*}{Theorem 6.20}
\newtheorem*{theorem621*}{Theorem 6.21 Fundamental Theorem of Calculus}
\newtheorem*{theorem622*}{Theorem 6.22 Integration by Parts}    
\newtheorem*{theorem623*}{Theorem 6.23}
\newtheorem*{theorem624*}{Theorem 6.24}
\newtheorem*{theorem625*}{Theorem 6.25}
\newtheorem*{theorem627*}{Theorem 6.27}

\newtheorem*{def626*}{Definition 6.26 Rectifiable Curve}                


\begin{document}

\begin{def231*} $ \newline $
Let $ X $ be a metric space. Let $ E \subseteq X $. Let $ \left\{G_{\alpha}\right\} \subseteq X $ be an indexed set family, the elements of which are open subsets of $ X $. $ \left\{ G_{\alpha}\right\} $ is an open cover of $ E $ when:
$$ E \subseteq \bigcup_{\alpha} G_{\alpha} $$
\end{def231*}




\begin{def232*} $ \newline $
Let $ X $ be a metric space. Let $ K \subseteq X $. $ K $ is compact if every open cover of $ K $ contains a finite subcover. That is, $ K $ is compact when the existence of an open cover implies the existence of a finite subcover:
$$ K \subseteq \bigcup_{\alpha \in A} G_{\alpha} \implies \exists m : K \subseteq \bigcup_{j=1}^{m} G_{\alpha_{j}} $$  
\end{def232*}

$\newline$
\begin{def418*} $\newline$
Let $ X, Y $ be metric spaces. Let $ f : X \to Y $. $ f $ is uniformly continuous on $ X $ when:
$$ \forall \epsilon > 0, \exists \delta > 0, \forall p, q \in X : d_{X}\left(p,q\right) < \delta \implies d_{Y}\left( f\left(p\right), f\left(q\right)\right) < \epsilon $$
\end{def418*}

$\newline$
\begin{theorem423*} $\newline$
Let $ f : \left[a,b\right] \to \mathbb{R} $ be continuous. \\
If:
\begin{align*}
& 1. \hspace{.75 em} f\left(a\right) < f\left(b\right) \\
& 2. \hspace{.75 em} \exists c \in \mathbb{R} : f\left(a\right) < c < f\left(b\right)
\end{align*}
Then:
$$ \exists x \in \left(a,b\right) : c = f\left(x\right) $$
\end{theorem423*}


$\newline$
\begin{theorem419*}
Let $ X, Y $ be metric spaces, and assume $ X $ is compact. Let $ f : X \to Y $ be a continuous mapping. Then $ f $ is uniformly continuous on $ X $.
\end{theorem419*}

$\newline$
\begin{def61*} Let $ \left[a,b\right] \subseteq \mathbb{R} $. A partition of $ \left[a,b\right]$ is a finite set of points $ \left\{x_{0}, \ldots, x_{n}\right\} $, where
$$ a = x_{0} \leq x_{1} \leq \ldots \leq x_{n} = b \quad ; \quad \Delta x_{i} = x_{i} - x_{i-1} $$
Let $ f : \left[a,b\right] \to \mathbb{R} $ be a bounded function. To each partition $ P $ of $ \left[a,b\right] $, we define:

\begin{align*}
& U\left(P,f\right) = \sum_{i=1}^{n} M_{i} \Delta x_{i} \quad ; \quad M_{i} = \text{\rm{sup}} \left\{ f\left(x\right) : x \in \left[x_{i-1},x_{i}\right] \right\} \\
& L\left(P,f\right) = \sum_{i=1}^{n} m_{i} \Delta x_{i} \quad ; \quad m_{i} = \text{\rm{inf}} \left\{ f\left(x\right) : x \in \left[x_{i-1},x_{i}\right] \right\} \\
& \overline{\int_{a}^{b}} f d\alpha = \text{\rm{inf }} U\left(P,f\right) \\
& \underline{\int_{a}^{b}} f d\alpha = \text{\rm{sup }} L\left(P,f\right) 
\end{align*}
\end{def61*}


$\newline$
\textbf{Function Assumptions for Stieljes Integral}
\begin{align*}
& \text{\rm{1. }} f : \left[a,b\right] \to \mathbb{R}  \\
& \text{\rm{2. }} f\text{\rm{ is bounded }} \\
& \text{\rm{3. }} \alpha : \left[a,b\right] \to \mathbb{R} \text{\rm{ is monotonically increasing }} \\
\end{align*}

$\newline$
\textbf{Definition: Riemann Integrable Function} \\
Let $ f : \left[a,b\right] \to \mathbb{R} $, and let $ P = \left\{a = x_{0} \leq x_{1} \leq \ldots \leq x_{n} = b \right\} $. $ f $ is Riemann integrable when its lower and upper Riemann integrals are equal:
$$ f \in \mathscr{R} \Longleftrightarrow \underline{\int_{a}^{b}} f d\alpha = \overline{\int_{a}^{b}} f d\alpha = \int_{a}^{b} f d\alpha $$


$\newline$
\begin{def63*} $\newline$
Let $ P $ be a partition. $ P^{*} $ refines $ P $ if every point of $ P $ is also a point of $ P^{*} $:
$$ P^{*} : P \subseteq P^{*} $$
\end{def63*}

$\newline$
\textbf{Definition: Common Refinement of Multiple Partitions} \\
Let $ P_{1} $ and $ P_{2} $



$\newline$
\begin{theorem64*} If $ P^{*} $ is a refinement of $ P $, then:
$$ L\left(P, f, \alpha\right) \leq L\left(P^{*}, f, \alpha\right) \quad\text{\rm{ and }} \quad U\left(P^{*}, f, \alpha\right) \leq U\left(P,f,\alpha\right) $$
\end{theorem64*}

$\newline$
\begin{theorem65*} $ \underline{ \int_{a}^{b} } f d\alpha \leq \overline{ \int_{a}^{b} } f d\alpha $ \end{theorem65*}

$\newline$
\begin{theorem66*} $ f \in \mathscr{R} $ on $ \left[a.b\right] $ if and only if for every $ \epsilon > 0 $, there exists a partition $ P $ such that:
$$ U\left(P,f,\alpha\right) - L\left(P,f,\alpha\right) < \epsilon $$
\end{theorem66*}

$\newline$
\begin{theorem68*} If $ f $ is continuous on $ \left[a,b\right] $, then $ f \in \mathscr{R}\left(\alpha\right) $ on $ \left[a,b\right] $.
\end{theorem68*}


$\newline$
\begin{theorem69*}
If $ f $ is monotonic on $ \left[a,b\right] $, and if $ \alpha $ is continuous on $ \left[a,b\right] $, then $ f \in \mathscr{R}\left(\alpha\right) $.
\end{theorem69*}



\begin{theorem610*}
Suppose $ f $ is bounded on $ \left[a,b\right]$, $ f $ is continuous almost everywhere on $ \left[a,b\right] $, and $ \alpha $ is continuous at every point at which $ f $ is discontinuous. Then $ f $ is Riemann integrable.
\end{theorem610*}


\begin{theorem611*}
Suppose $ f \in \mathscr{R}\left(\alpha\right) $ on $ \left[a,b\right] $, $ f \in \left[m,M\right] $, $ \phi $ is continuous on $ \left[m,M\right] $, and $ h\left(x\right) = \phi\left(f\left(x\right)\right) $ on $ \left[a,b\right] $. Then $ h \in \mathscr{R}\left(\alpha\right) $ on $ \left[a,b\right]$.
\end{theorem611*}

\newpage

\begin{theorem612*} $\newline$
(a) 
\begin{align*}
& f_{1} \in \mathscr{R}\left(\alpha\right)_{\left[a,b\right]} \land f_{2} \in \mathscr{R}\left(\alpha\right)_{\left[a,b\right]} \implies \left(f_{1} + f_{2}\right) \in \mathscr{R}\left(\alpha\right)_{\left[a,b\right]} \land \int_{a}^{b} \left(f_{1} + f_{2}\right) d\alpha = \int_{a}^{b} f_{1} d\alpha + \int_{a}^{b} f_{2} d\alpha \\
& f \in \mathscr{R}\left(\alpha\right)_{\left[a,b\right]} \implies \forall c \in \mathbb{R} : \left(cf\right) \in \mathscr{R}\left(\alpha\right)_{\left[a,b\right]} \land \int_{a}^{b} \left(cf\right) d\alpha = c \int_{a}^{b} f d\alpha
\end{align*}

\noindent (b)
\begin{align*}
& \forall x \in \left[a,b\right] : f_{1}\left(x\right) \leq f_{2}\left(x\right) \implies \int_{a}^{b} f_{1} d\alpha \leq \int_{a}^{b} f_{2} d\alpha
\end{align*}

\noindent (c) If $ f \in \mathscr{R}\left(\alpha\right)_{\left[a,b\right]} $, and $ c \in \left(a,b\right) $, then: \\
\begin{align*}
& \text{\rm{1. }} f \in \mathscr{R}\left(\alpha\right)_{\left[a,c\right]} \\
& \text{\rm{2. }} f \in \mathscr{R}\left(\alpha\right)_{\left[c,b\right]} \\
& \text{\rm{3. }} \int_{a}^{c} f d\alpha + \int_{c}^{b} f d\alpha = \int_{a}^{b} f d\alpha
\end{align*}


\noindent (d) $$ f \in \mathscr{R}\left(\alpha\right)_{\left[a,b\right]} \land \exists M \in \mathbb{R^{+}} : | f\left(x\right) | \leq M \implies \bigg\lvert \int_{a}^{b} f d\alpha \bigg\rvert \leq M \cdot \left(\alpha\left(b\right) - \alpha\left(a\right) \right)$$ \\
(e)
\begin{align*}
& f \in \mathscr{R}\left(\alpha_{1}\right) \land f \in \mathscr{R}\left(\alpha_{2}\right) \implies f \in \mathscr{R}\left(\alpha_{1} + \alpha_{2}\right) \land \int_{a}^{b} f d\left( \alpha_{1} + \alpha_{2} \right) = \int_{a}^{b} f d\alpha_{1} + \int_{a}^{b} f d\alpha_{2} \\
& \forall c \in \mathbb{R} : f \in \mathscr{R}\left(\alpha\right) \implies f \in \mathscr{R}\left(c\alpha\right) \land \int_{a}^{b} f d\left(c\alpha\right) = c \int_{a}^{b} f d\alpha
\end{align*}

\end{theorem612*}

\newpage

\begin{theorem613*} If $ f \in \mathscr{R}\left(\alpha\right) $ and $ g \in \mathscr{R}\left(\alpha\right) $ on $ \left[a,b\right] $, then:
\begin{align*}
& \text{\rm{1. }} fg \in \mathscr{R}\left(\alpha\right) \\
& \text{\rm{2. }} | f | \in \mathscr{R} \left(\alpha\right) \text{ and } \bigg\lvert \int_{a}^{b} f d\alpha \bigg\rvert \leq \int_{a}^{b} | f | d\alpha
\end{align*}

\end{theorem613*}

%\begin{equation*}
%F \left(x\right) := \left\{
%	\begin{array}{ll}
%		\int\limits_{a}^{x} f\left(t\right) dt  & ; \quad x \in \left[a,b\right] \vspace{.15 em} \\
%		0 & ; \quad \text{\rm{else}}
%	\end{array}
%\right.
%\end{equation*}
\begin{def614*} $ \newline $
\begin{equation}
I\left(x\right) = \left\{
	\begin{array}{ll}
		0 & ; \quad x \leq 0 \\
		1 & ; \quad x > 0
	\end{array}
\right.
\end{equation}
\end{def614*}

\begin{theorem615*} If $ a < s < b $, $ f $ is bounded on $ \left[a,b\right] $, $ f $ is continuous at $ s $, and $ \alpha\left(x\right) = I\left(x-s\right) $, then:
$$ \int_{a}^{b} f d\alpha = f\left(s\right) $$
\end{theorem615*}



\begin{theorem616*} $ \newline $
Suppose:
\begin{align*}
& \text{\rm{1. }} \forall n \in \mathbb{Z}^{+} : c_{n} \geq 0 \\
& \text{\rm{2. }} \sum c_{n} \text{\rm{ convergent }} \\
& \text{\rm{3. }} \left\{s_{n}\right\} \subseteq \left(a,b\right) \text{\rm{ sequence of distint points }} \\
& \text{\rm{4. }} \alpha\left(x\right) = \sum_{n=1}^{\infty} c_{n} \cdot I\left(x-s_{n}\right) \\
& \text{\rm{5. }} f \text{ \rm{ is continuous on }} \left[a, b\right]
\end{align*}
Then:
$$ \int_{a}^{b} f d\alpha = \sum_{n=1}^{\infty} c_{n} \cdot f\left(s_{n}\right) $$
\end{theorem616*}

\begin{theorem617*} $ \newline $
Suppose:
\begin{align*}
& \text{ \rm{1. }} \alpha \text{\rm{ monotonic nondecreasing }} \\
& \text{ \rm{1. }} \alpha' \in \mathscr{R} \text{\rm{ on }} \left[a,b\right] \\
& \text{ \rm{3. }} f : \left[a,b\right] \to \mathbb{R} \text{\rm{ bounded }}
\end{align*}
Then:
$$ f \in \mathscr{R} \left(\alpha\right) \Longleftrightarrow f\alpha' \in \mathscr{R} $$
Consequently, we write:
$$ \int_{a}^{b} f d\alpha = \int_{a}^{b} f\left(x\right) \alpha'\left(x\right) dx $$
\end{theorem617*}



\begin{theorem622*}
Suppose $ F $ and $ G $ are differentiable functions on $ \left[a,b\right] $, $ F' = f, G' = g $. Assume $ f \in \mathscr{R} $ and $ g \in \mathscr{R} $. Then:
$$
\int_{a}^{b} F\left(x\right) g\left(x\right) dx = F\left(b\right)G\left(b\right) - F\left(a\right)G\left(a\right) - \int_{a}^{b} f\left(x\right) G\left(x\right) dx
$$

\end{theorem622*}


















\end{document}